\report
\fontfam[Latin Modern]
\enlang
\enquotes
\let\_firstnoindent=\relax
\activettchar`

\def\LLTeX{LL\TeX}
\def\Metapost{Metapost}
\def\eTeX{\leavevmode\hbox{$\varepsilon$}-\TeX}

\tit \LLTeX

\author Michal Vlasák, 2020

\sec Introduction

\LLTeX\ is a lightweight \LuaTeX\ distribution for Linux. It aims to provide
minimalist and modern base for Plain \LuaTeX\ and \OpTeX\ formats, but doesn't
limit their potential.

\LLTeX\ was created as a semestral work for the purposes of BI-TEX course at
FIT CTU in Prague. It's repository is hosted at
\url{https://github.com/lahcim8/lltex/}.

\sec Motivation

I have been dissatisfied with TeX Live. It is complicated and huge. The full
scheme installation, which is the default, takes $7\,075\,\hbox{MiB}$. Even the
minimal scheme which only includes Plain \TeX\ is 106\,MiB in size. There are
options to exclude source files and documentation which reduce size of full
scheme to 3\,899\,MiB, but that is still a lot.

As a monolith it also doesn't really fit very much into a Unix system. It's
default locations of `TEXMF` trees are not complaint to `hier(7)`. Instead,
everything is in an isolated tree.

I also wanted to explore how things really worked and what is really needed to
get Plain running, while not missing any of \LuaTeX\ features.

\sec Target users

We target users who want to easily install a small ($\approx 32\,\hbox{MiB}$),
but very functional \TeX\ system. As \LLTeX\ includes \OpTeX\ it can even serve
as a replacement of \LaTeX\ and it's packages.

We also target use in Docker images where the small size matters too. While
being able to install \LLTeX\ from a Linux distribution package manager makes
it easy to use.

\sec \LuaTeX\ as the only \TeX\ engine

Historically there have been a few \TeX\ engines. Each brought it's advancement
over Knuth's original `tex`. The one which has had most development over the
last years is \LuaTeX. It's distinct aspect is that a Lua language interpreter
is baked into it. A lot of aspects of how \TeX\ works can now be influenced or
completly changed with Lua code. Another nice feature is integrated \Metapost\
in the form of `mplib`.

But for a \TeX\ outsider the main benefits are that it outputs PDF files, fully
supports Unicode and can use OpenType fonts. There aren't many reasons for not
using \LuaTeX\ for new documents. That is why \LLTeX\ includes only this
enginge.

\sec \OpTeX\ and Plain \LuaTeX\ formats

The first choice for a format was Plain \TeX\ or rather it's adaptation for
\LuaTeX. Plain has what most users would need anyway, and has been a base for
other formats. One of them is \OpTeX\ which is a \LuaTeX\ only format. It keeps
the simplicity of Plain \TeX\ while in a sense offers even more then \LaTeX\
--- where \LaTeX\ needs a package or external binary, \OpTeX\ has it built in.

As the dependencies of these two formats are mostly the same they are both
included in \LLTeX\ without much of an overhead.

\sec Language support

Previous \TeX\ engines had the limitation of being able to load hyphenation
patterns only at format creation time --- when running ini\TeX. \LuaTeX\ has no
such limitation and by using Lua it is possible to load hyphenation patterns at
any time.

Nowadays virtually all hyphenation patterns that have been used by \TeX\ users
are included in the `hyph-utf8` CTAN package. All patterns are provided in the
form usable by older \TeX\ enginges (which had to deal with 8-bit font
encodings) and the new plain text patterns directly loadable by \LuaTeX.

To save space and to not include hyphenation patterns of all languages in the
format, TeX Live provides hyphenation patterns for each language in a different
package. Each package then registers itself using the TeX Live \"execute
AddHyphen". An example for Czech language:

\begtt
execute AddHyphen \
	name=czech \
	lefthyphenmin=2 \
	righthyphenmin=3 \
	file=loadhyph-cs.tex \
	file_patterns=hyph-cs.pat.txt \
	file_exceptions=hyph-cs.hyp.txt
\endtt

This information is written to the `language.dat`, `language.def` and
`language.dat.lua` files. Only the last two are used by \eTeX\ language
mechanism, which is used by Plain \LuaTeX.

This is what the above `AddHyphen` directive adds to `language.def`:

\begtt
\addlanguage{czech}{loadhyph-cs.tex}{}{2}{3}
\endtt

and this is what gets added to `language.dat.lua`:

\begtt
	['czech'] = {
		loader = 'loadhyph-cs.tex',
		lefthyphenmin = 2,
		righthyphenmin = 3,
		synonyms = {  },
		patterns = 'hyph-cs.pat.txt',
		hyphenation = 'hyph-cs.hyp.txt',
	},
\endtt


Normally `language.def` is read by `etex.src` at format creation time and for
listed language registers them and loads their hyphenation patterns. That
enables their use later with `\uselanguage`.

As in \LuaTeX\ it is even discouraged to load patterns into format, the
mechanism is changed by `hyph-utf8`'s own `etex.src`. Instead of loading the
patterns on `\addlanguage` the language is only registred and the patterns are
loaded on the first `\uselanguage`. Both `\addlanguage` and `\uselangauge`
actually use `luatex-hyphen.lua`, which uses information in `language.dat.lua`
for handling synonyms and finding the names of pattern files (other fields in
`language.dat.lua` are unused).

\LLTeX\ includes all hyphenation patterns in the `txt` format --- they take up
only 260\ KiB afterall. To enable the use of all of them a `language.def` file
is generated, which allows `etex.src` to register all languages. A minimal
`language.dat.lua` file is also generated and contains the file names of
pattern files of all supported languages.

In \OpTeX\ the situation is simpler as it already includes the information that
\eTeX\ expects in  `language.def`. However it still uses `luatex-hyphen.lua`
and `language.dat.lua`.

\sec Fonts

To use \LuaTeX\ to it's full potential it is better to use OpenType fonts.
These are the same fonts that are used by other programs and as such some of
them are already preinstalled on operating systems. And probably many more are
additionally installed by users or administrators.

Traditionally Computer Modern family of fonts has been used as the default for
\TeX. There are OpenType adaptations of Computer Modern --- most notably Latin
Modern. But Latin Modern fonts are often already packaged for many systems. For
example there is `fonts-lmodern` in Ubuntu and `otf-latin-modern` in Arch
Linux. As \LLTeX\ aims to integrate well into these systems it doesn't provide
any OpenType fonts and instead expects the user to install fonts in the
standard ways. \LLTeX\ can easily find them and use them.

\secc 8-bit fonts

Only 8-bit fonts can be preloaded into a \TeX\ format. Both \OpTeX\ and Plain
do it. To support this \LLTeX\ includes minimal set of Type 1 fonts and their
respective metric and encoding files. A minimal `pdftex.map` file was created
which is used to map names of `.tfm` metric files, to font names and font
files, with optional reencodings. As this file isn't supposed to be used by
`dvips` it's directives are not included. This is an example from `pdftex.map`:

\begtt
cmr5 CMR5 <cmr5.pfb
\endtt

This line makes connection between cmr5.tfm metric file, cmr5.pfb Type 1 font
file and CMR5 font name. (CMR5 stands for Computer Modern Roman in 5 point
optical size).

Another example:

\begtt
ec-lmr5 LMRoman5-Regular <lm-ec.enc <lmr5.pfb
\endtt

This is similiar as the previous case but additionally uses the encoding vector
stored in file `lm-ec.enc`. This is necessary because `lmr5.pfb` actually
contains many glyphs, while \TeX\ can use only 256 of them and expects the
order to correspond with `ec-lmr5.tfm` which contains metric information for
those selected 256 glyphs. In this particular case the Cork (\"EC") encoding is
used.

\secc OpenType fonts

In order to handle OpenType fonts some Lua code is needed. `luaotfload` is
included for this purpose as it is already used internally by \OpTeX. It can
also be used from Plain \LuaTeX, by using `\input luaotfload.sty`.

\sec `luamplib`

\LLTeX\ includes `luamplib` package which provides user level macros for using
`mplib`. `mplib` is \Metapost{} in form of a library built into \LuaTeX. To use
it: `\input luamplib.sty`.

\sec Finding files

\LuaTeX\ uses the `kpathsea` library for finding files. `kpathsea` uses path
specifications and variables similiar to the `PATH` environment variable. But
different variables are used for different file types. For example the variable
`TEXINPUTS` determines the location of files for `\input` (most likely `.tex`
files). These variables can be set from (in order of significance):

\begitems
* environment variables,
* `texmf.cnf` files,
* defaults set at compilation.
\enditems

For finding `texmf.cnf` files the same path searching mechanism is used (but
the location cannot be set from configuration file). This time the variable
`TEXMFCNF` is used. All found `texmf.cnf`'s are taken into account. Earlier
assignments in configuration files override later ones. That's why in \LLTeX\
we compile \LuaTeX\ such that it first looks for `texmf.cnf` first in
`$XDG_CONFIG_HOME/lltex/texmf.cnf` (where `$XDG_CONFIG_HOME` defaults to
`~/.config`) for user configuration, then in `/etc/lltex/texmf.cnf`, for system
wide configuration. The last possible llocation is set to \LLTeX's own config
file. This order essentially means that the user can always change anything.

The fallback \LLTeX\ `texmf.cnf` resides in `../share/lltex/web2c/texmf.cnf`
(relative to `luatex` binary). It sets `TEXMFVAR` (for cache files) and
`OSFONTDIR` (for font files) to locations expected on Unix systems, so
everything should work out of the box.

We also use the variable `TEXMFDOTDIR` (which defaults to `.`) as a primary
search path for every file type. This can be useful for temporary overrides on
the command line, and was inspired by TeX Live.

All other interesting variables are set so they are complain with the \"\TeX\
Directory Structure" standard. All roots of TDS trees are expected in the
`TEXMF` variable, which as always can be overriden by user and defaults to our
own tree (`_TEXMFMAIN`), user tree `TEXMFUSER` and system tree `TEXMFSYSTEM`.
The order is important again here. We look first amongst our files because they
are important for basic functionality and users most likely don't wont to
override them (and if they do, they can use `TEXMFDOTDIR`).

\sec Building \LLTeX

\LLTeX\ is easily packagable as one of it's goals is to integrate well into
Unix systems. As an example of that the implementation of \LLTeX\ is really a
`PKGBUILD` file which contains metadata used by `makepkg` program to create
Arch Linux package. To accomodate other distributions a `build-deb` script is
provided, which uses information in `PKGBUILD` to create a `.deb` package,
which should be usable by a large part of Linux users.

\secc[build] Build instructions

To build Arch Linux package run in the root of \LLTeX\ repository:

\begtt
makepkg
\endtt

To build `.deb` package, run:

\begtt
./build-deb
\endtt

To build both packages without rebuilding everything twice, you can:

\begtt
makepkg
env USE_MAKEPKG_PKG=1 ./build-deb
\endtt

\sec Installing \LLTeX

\secc Arch Linux

After obtaining `.pkg.tar.zst` file (see Section~\ref[build]), you can install it by executing:

\begtt
pacman -U lltex-0.1-1-x86_64.pkg.tar.zst
\endtt

\secc Debian-based systems

After obtaining `.deb` file (see Section~\ref[build]), you can install it by executing:

\begtt
dpkg -i lltex_0.1-1_amd64.deb
\endtt

\bye
